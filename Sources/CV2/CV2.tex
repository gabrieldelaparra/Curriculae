% CV fictif destiné à servir de modèle à la classe de curriculum vitae cv.cls
% Le fichier cv.cls doit se trouver dans le même répertoire ou dans un répertoire
% accessible par LaTeX (voir l'utilisation de la variable TEXINPUTS).
% 5 Février 2003 -- Frédéric Meynadier (Frederic.Meynadier@obspm.fr)
%
%
\documentclass{cv}

\usepackage[francais]{babel} 
\usepackage[latin1]{inputenc}  %% les accents dans le fichier.tex
\usepackage[T1]{fontenc}       %% Pour la césure des mots accentués
\usepackage[paper=a4paper,textwidth=160mm,twosideshift=0pt]{geometry}

\newcommand{\lieu}[1]{{#1}\ }
\newcommand{\activite}[1]{\textbf{#1}\ }
\newcommand{\comment}[1]{\textsl{#1}\ }



\begin{document}

\begin{chapeau}
\begin{adresse}
	Isaac NEWTON\\%
	9.81, rue des Pommiers\\%
	Trinity College, Londres\\%
	\ligne\\%
	Tél. : 06 67 25 90 00\\%
	E-mail : \texttt{inewton@apple.com}
\end{adresse}
\begin{etatcivil}
	Né le 25/12/1642\\
	Nationalité Anglaise
\end{etatcivil}
\end{chapeau}



	%%%%%%%%%%%%%%%%%%
	% Bloc rubriques %
	%%%%%%%%%%%%%%%%%%

\begin{rubriquetableau}[3.5cm]{Formation}

1665--1669 
	& \activite{Recherches à domicile}
	\comment{mention Très Bien}
	\lieu{Lincolnshire}\\

1661--1665 
	& \activite{B.A. Degree, Mathématiques}
	\lieu{Université de Cambridge}\\

\end{rubriquetableau}

\begin{rubriquetableau}[3.5cm]{Activités Professionnelles}
1673--1683
        & \activite{Enseignement de l'Algèbre}
        \lieu{(Trinity College, Cambridge)}\\
1665--1666
	& \activite{Observation de la chute des pommes}
	\lieu{Verger familial}\\


\end{rubriquetableau}

\begin{rubrique}{Langues} 
Anglais courant.
\end{rubrique}

\begin{rubrique}{Compétences en informatique}%
Système : Linux

Langage : Pascal
\end{rubrique}


\end{document}






