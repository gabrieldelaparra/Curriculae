%% Exemple de CV en LaTeX. 
%% Nicolas Couchoud
%% 2000

\documentclass[12pt,a4paper]{article}
\usepackage[francais]{babel}
\usepackage[latin1]{inputenc}

\begin{document}

% Je ne veux pas de numero de page
\pagestyle{empty}

% \annee est la largeur de la premiere colonne, c'est a dire celle
% contenant l'annee scolaire. Elle est ici definie comme etant la
% largeur du texte « janvier-fevrier ». À adapter le cas echeant.

\newlength{\annee}
\settowidth{\annee}{Janvier--fevrier}

% \texte est la largeur de la deuxieme colonne. Elle est definie comme
% etant la largeur de la page moins celle de la premiere colonne.
% 2\tabcolsep est la largeur de l'espacement entre les colonnes.

\newlength{\texte}
\setlength{\texte}{\textwidth} \addtolength{\texte}{-\annee} 
	\addtolength{\texte}{-2\tabcolsep}

\begin{center} \large \sc Curriculum vitae \end{center}

% Le \noindent au debut et les \\ ensuite servent a eviter
% l'indentation. (Idem dans la rubrique « Divers ».)
\noindent {\large Nicolas \sc Couchoud} \\
48, boulevard Jourdan \\
75014 Paris \\
Tel. : 01.45.80.76.04

% Ici vient le CV lui-même.
% Les @{} servent a eviter que LaTeX mette un espacement avant la premiere
% colonne et apres la derniere.
% Les \par servent a passer a la ligne au sein d'une colonne.

\subsection*{Estudios}
\noindent \begin{tabular}{@{}p{\annee}p{\texte}@{}}
 & Ne le 12 janvier 1978 a Saint-Etienne (Loire). \\
1994 & Baccalaureat C (mathematiques et sciences physiques), mention 
tres bien. \\
1994--1996 & Mathematiques Superieures et Mathematiques Speciales au lycee
Claude Fauriel (Saint-Etienne). \\
1996 & Entree a l'Ecole Normale Superieure de Paris. \\
1996--1997 & Licence de physique, mention bien et debut de maîtrise a
l'Universite Paris VI. \par
	Magistere interuniversitaire de physique (MIP) premiere annee. \\
1997--1998 & Fin de maîtrise de physique, mention bien. \par
	Magistere interuniversitaire de physique (MIP) deuxieme annee. \\
1998--1999 & DEA de Physique Theorique a l'ENS, mention tres bien 
(3eme rang). \\
1999--2000 & Preparation a l'agregation de sciences physiques, option physique.
\par
	Fin de la troisieme annee du magistere interuniversitaire de physique.
\end{tabular}

\subsection*{Stages en laboratoire}
\noindent \begin{tabular}{@{}p{\annee}p{\texte}@{}}
Septembre 1997 & Etude de la force de van der Waals entre des atomes de
cesium et une paroi. \par
	Laboratoire de Physique des Lasers, Universite Paris XIII. \\
Janvier--juin 1998 & Mesure de la masse du neutrino electronique. \par
	Universite de Mayence (Allemagne). \\
Janvier--fevrier 1999 & Photoproduction de mesons dans l'accelerateur
CEBAF. \par
	Commissariat a l'Energie Atomique (Saclay).
\end{tabular}

\subsection*{Divers}
\noindent
Anglais et allemand courants. \\
Connaissance de base Unix (dont Linux). \\
Tuteur informatique a l'ENS depuis janvier 1999. (Les tuteurs sont des eleves
volontaires qui encadrent des stages de formation aux machines Unix.) \\
Connaissance de base du langage C.

\end{document}
